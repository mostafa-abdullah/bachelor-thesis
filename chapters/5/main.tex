% !TeX root = ../main.tex
% Add the above to each chapter to make compiling the PDF easier in some editors.

\chapter{Evaluation}\label{chapter:eval}
Having explained the encoding for the type system in Z3 and the type inference rules and the constraints generated by different Python constructs, we discuss here the experimentation we have done for the tool and discuss some of the current type inference limitations.

\section{Experimentation}
Testing the type inference was done by giving the tool a variety of programs, some of which focus on a single functionality (like multiple inheritance, function calls, etc.), while other are open source code gathered from different online open source platforms. However, all the open source projects that we use in our experimentation was not meant to be statically typed when it was written, so some parts of these projects oppose the restrictions imposed by having a nominal static type system for Python. An example of these parts is the following:

\begin{lstlisting}
class A:
	pass
	
class B(A):
	def f(self): ...

class C(A):
	def f(self): ...

def foo(x):
	x.f()
	
f(B())
f(C())
\end{lstlisting}

The type of the argument \lstinline|x| in function \lstinline|foo| is inferred to be of type \lstinline|A| (The super type of both \lstinline|B| and \lstinline|C|). However, class \lstinline|A| does not implement the method \lstinline|f|, so the call \lstinline|x.f()| in the body of \lstinline|foo| is invalid, although it will not fail at runtime.\\

Accordingly, we had to modify some of the projects we used to fit the limitations imposed by having a static type system. \\

After the types of these programs are inferred and the typed source code is generated, we run mypy \cite{mypy} to statically check these types and verify that the inference is correct.
\subsection{IMP Interpreter}
IMP \cite{imp} is a simple programming language developed in the 1970s. An interpreter for the language \cite{imp_i} was created by Jay Conrod as an example of building interpreters. This interpreter is an excellent testing material for our type inference for many reasons:

\begin{itemize}
	\item It does not violate any of the restrictions discussed in \ref{sub:st_ts_3}.
	\item It uses most of the Python patterns that we support, like intensive inheritance, callable objects, operator overloading and using built-in libraries.
	\item It is composed of more than 1000 lines of code, which is comparable to most Python projects.
\end{itemize}

The inference for this project runs in [...] seconds, which gives a prospect that the performance of the constraints solving is capable of handling a large portion of sized projects.

\section{Limitations}